\documentclass[12pt]{article}

\usepackage[utf8x]{inputenc}
\usepackage[T1, T2A]{fontenc}
\usepackage{fullpage}
\usepackage{multicol,multirow}
\usepackage{tabularx}
\usepackage{ulem}
\usepackage{listings} 
\usepackage[english,russian]{babel}
\usepackage{tikz}
\usepackage{pgfplots}
\usepackage{indentfirst}
\usepackage{ulem} 


\parindent=1cm
\makeatletter
\newcommand{\rindex}[2][\imki@jobname]{%
    \index[#1]{\detokenize{#2}}%
}
\makeatother
\newcolumntype{P}[1]{>{\raggedbottom\arraybackslash}p{#1}}

\linespread{1}
\pgfplotsset{compat=1.16}
\begin{document}

\section*{\centering Лабораторная работа №0 по курсу:\\ Операционные системы}

\noindent Выполнил студент группы М8О-208Б-17 МАИ \,\, \textit{Милько Павел}.

\subsection*{Цель работы}

Получение навыков использования утилиты strace.



\subsection*{Задание}


{\bf Часть 1.} Написать собственную программу, которая демонстрирует работу с различными вызовами (8-15) операционной системы. Произвести диагностику
работы написанной программы с помощью утилит ОС, изучив основные принципы
применения используемых утилит.

{\bf Часть 2. }Выбрать стороннее программное обеспечение. Произвести диагностику ПО.
Выявить ключевые особенности работы. Выявить предполагаемые ключевые системные вызовы, которые используются в стороннем программном обеспечении.

\subsection*{Информация}

Системные вызовы -- один из уровней абстракции, предоставляемый пользовательским программам. Само управление ресурсами проходит незаметно для пользователя и осуществляется в автоматическом режиме. Любой однопроцессорный компьютер одномоментно может выполнить только одну команду. Когда процесс выполняет
пользовательскую программу в режиме пользователя и нуждается в некоторой
услуге ОС, он должен выполнить команду системного прерывания, чтобы передать
управление ОС. Затем ОС по параметрам вызова определяет, что именно требуется
вызывающему процессу. После этого она обрабатывает системный вызов и возвращает управление той команде, которая следует за системным вызовом. В некотором
смысле выполнение системного вызова похоже на выполнение особой разновидности
вызова процедуры, с той лишь разницей, что системные вызовы входят в ядро, а
процедурные -- нет.
Использованные системные вызовы:


\subsection*{Использованные системные вызовы}

\begin{enumerate}
\item \lstinline[language=C]|int creat(const char *pathname, modet mode);| -- создает и открывает
файловый дескриптор в соответствии с флагами открытия.
\item \lstinline[language=C]|int open(const char *path, int oflag, ...);| -- открывает файловый дескриптор: первый аргумент -- путь до файла, второй -- флаги открытия. */
\item \lstinline[language=C]|int read(int fd, void *buffer, int nbyte);| -- читает nbyte из файлового дескриптора fd в буффер buffer.
\item \lstinline[language=C]|int write(int fd, void *buffer, int nbyte);| -- записывает количество байтов
в 3 аргументе из буфера в файл с дескриптором fd, возвращает количество
записанных байтов или -1 в случае ошибки.
\item \lstinline[language=C]|int close(int fd);| -- закрывает файловый дескриптор.
\item \lstinline[language=C]|int fsync(int fd);| -- синхронизирует состояние файла в памяти с его состоянием
на диске
\item \lstinline[language=C]|int truncate(int fd, offt lenght);| -- устанавливает длину файла с файловым
дескриптором fd в lenght байт. Если файл до этой операции был длиннее, то
отсеченные данные теряются. Если файл был короче, то он увеличивается, а
добавленная часть заполняется нулевыми байтами.
\item \lstinline[language=C]|pidt fork(void);| -- создает дочерний процесс. Если возвращает 0, то созданный
процесс -- ребенок, если >0, то -- родитель.
\item  \lstinline[language=C]|pidt wait(int *status);| -- приостанавливает выполнение текущего процесса
до тех пор, пока дочерний процесс не завершится, или до появления сигнала, который либо завершает текущий процесс, либо требует вызвать функцию-обработчик.
\item \lstinline[language=C]|offt lseek(int fd, offt offset, int whence);| -- устанавливает смещение для файлового дескриптора в значение аргумента offset в соответствии с директивой
whence, которая может принимать одно из следующих значений: SEEKSET(смещение устанавливается в offset байт от начала файла), SEEKCUR(смещение устанавливается как текущее смещение плюс offset байт), SEEKEND(смещение устанавливается как размер файла плюс offset байт).
\item \lstinline[language=C]|int execl(const char *path, const char *arg);| -- передаем аргументы в командную строку (когда их количество известно)
\item \lstinline[language=C]|void exit(int status);| --выходит из процесса с заданным статусом.

\end{enumerate}


\subsection*{Исходный код}

{
\footnotesize
Лабораторная работа 2:
\lstinputlisting[language=C,escapechar=~]{../../lab2/main.c}

Лабораторная работа 3:
\lstinputlisting[language=C,escapechar=~]{../../lab3/main.c}

Лабораторная работа 4:
\lstinputlisting[language=C,escapechar=~]{../../lab04/main.c}

Лабораторная работа 5:
\lstinputlisting[language=C,escapechar=~]{../../lab05/main.c}
\subsection*{Диагностика программ с помощью strace}

Лабораторная работа 2:
{
\scriptsize

\begin{lstlisting}[escapechar=!]
execve("./main", ["./main"], 0x7fff97f0d4a0 /* 48 vars */) = 0
./lab01 10
10.12.63	4d4691e7cfebe79b9722e274750b4222b2470c5433e8a15dbaac06b3
0.12.83	4ea299d7ad460e7b355e60fb04b2cd99ff36b36585c6d2d6eaf87f
3.10.125	42efe7b30ef5777e4cd4b1b927f672
15.12.246	4049b4392bc0998998b375
10.10.422	80c751cea9992cd8560624ca306ebb98
11.7.475	59ce7e1d0247467518
6.0.499	1c6b85f4fce7
29.9.1096	a4c44cd3bfb6438db30d88e570bf3d87c942edcae11de5f538de12975fd385
22.7.1321	
2.8.1994	e037ae90587681c36ad18b10028d4b6359
--- SIGCHLD {si_signo=SIGCHLD, si_code=CLD_EXITED, si_pid=22281, si_uid=1000, 
si_status=0, si_utime=0, si_stime=0} ---
+++ exited with 0 +++
\end{lstlisting}
}
Лабораторная работа 3:
{
\scriptsize

\begin{lstlisting}[escapechar=!]
Max count of threads: 5
Number of points: 10
Segment: [1,10]

clone(child_stack=0x7f319afbcfb0,
 flags=CLONE_VM|CLONE_FS|CLONE_FILES|CLONE_SIGHAND|CLONE_THREAD|CLONE_SYSVSEM|
CLONE_SETTLS|CLONE_PARENT_SETTID|CLONE_CHILD_CLEARTID, 
parent_tidptr=0x7f319afbd9d0, tls=0x7f319afbd700, child_tidptr=0x7f319afbd9d0) = 27737
clone(child_stack=0x7f319a7bbfb0,
 flags=CLONE_VM|CLONE_FS|CLONE_FILES|CLONE_SIGHAND|CLONE_THREAD|CLONE_SYSVSEM|
CLONE_SETTLS|CLONE_PARENT_SETTID|CLONE_CHILD_CLEARTID, 
parent_tidptr=0x7f319a7bc9d0, tls=0x7f319a7bc700, child_tidptr=0x7f319a7bc9d0) = 27738
clone(child_stack=0x7f3199fbafb0,
 flags=CLONE_VM|CLONE_FS|CLONE_FILES|CLONE_SIGHAND|CLONE_THREAD|CLONE_SYSVSEM|
CLONE_SETTLS|CLONE_PARENT_SETTID|CLONE_CHILD_CLEARTID,
 parent_tidptr=0x7f3199fbb9d0, tls=0x7f3199fbb700, child_tidptr=0x7f3199fbb9d0) = 27739
clone(child_stack=0x7f31997b9fb0,
 flags=CLONE_VM|CLONE_FS|CLONE_FILES|CLONE_SIGHAND|CLONE_THREAD|CLONE_SYSVSEM|
CLONE_SETTLS|CLONE_PARENT_SETTID|CLONE_CHILD_CLEARTID,
 parent_tidptr=0x7f31997ba9d0, tls=0x7f31997ba700, child_tidptr=0x7f31997ba9d0) = 27740
clone(child_stack=0x7f3198fb8fb0,
 flags=CLONE_VM|CLONE_FS|CLONE_FILES|CLONE_SIGHAND|CLONE_THREAD|CLONE_SYSVSEM|
CLONE_SETTLS|CLONE_PARENT_SETTID|CLONE_CHILD_CLEARTID, 
parent_tidptr=0x7f3198fb99d0, tls=0x7f3198fb9700, child_tidptr=0x7f3198fb99d0) = 27741
f(1.0000000000) = 1.38177329067603626989
clone(child_stack=0x7f319afbcfb0,
 flags=CLONE_VM|CLONE_FS|CLONE_FILES|CLONE_SIGHAND|CLONE_THREAD|CLONE_SYSVSEM|
CLONE_SETTLS|CLONE_PARENT_SETTID|CLONE_CHILD_CLEARTID,
 parent_tidptr=0x7f319afbd9d0, tls=0x7f319afbd700, child_tidptr=0x7f319afbd9d0) = 27742
f(2.0000000000) = 0.68142231392800700629
clone(child_stack=0x7f319a7bbfb0,
 flags=CLONE_VM|CLONE_FS|CLONE_FILES|CLONE_SIGHAND|CLONE_THREAD|CLONE_SYSVSEM|
CLONE_SETTLS|CLONE_PARENT_SETTID|CLONE_CHILD_CLEARTID, 
parent_tidptr=0x7f319a7bc9d0, tls=0x7f319a7bc700, child_tidptr=0x7f319a7bc9d0) = 27743
f(3.0000000000) = -0.57844065537114641717
clone(child_stack=0x7f3199fbafb0,
 flags=CLONE_VM|CLONE_FS|CLONE_FILES|CLONE_SIGHAND|CLONE_THREAD|CLONE_SYSVSEM|
CLONE_SETTLS|CLONE_PARENT_SETTID|CLONE_CHILD_CLEARTID, 
parent_tidptr=0x7f3199fbb9d0, tls=0x7f3199fbb700, child_tidptr=0x7f3199fbb9d0) = 27744
f(4.0000000000) = 0.81585937580395384572
clone(child_stack=0x7f31997b9fb0,
 flags=CLONE_VM|CLONE_FS|CLONE_FILES|CLONE_SIGHAND|CLONE_THREAD|CLONE_SYSVSEM|
CLONE_SETTLS|CLONE_PARENT_SETTID|CLONE_CHILD_CLEARTID, 
parent_tidptr=0x7f31997ba9d0, tls=0x7f31997ba700, child_tidptr=0x7f31997ba9d0) = 27745
f(5.0000000000) = -0.55091890659871411984
clone(child_stack=0x7f3198fb8fb0,
 flags=CLONE_VM|CLONE_FS|CLONE_FILES|CLONE_SIGHAND|CLONE_THREAD|CLONE_SYSVSEM|
CLONE_SETTLS|CLONE_PARENT_SETTID|CLONE_CHILD_CLEARTID, 
parent_tidptr=0x7f3198fb99d0, tls=0x7f3198fb9700, child_tidptr=0x7f3198fb99d0) = 27746
f(6.0000000000) = -0.37504068371550630667
f(7.0000000000) = -0.52914072009899415505
f(8.0000000000) = -0.44651974239025671309
f(9.0000000000) = 0.22281860995630112243
f(10.0000000000) = -0.30817877947797533977
+++ exited with 0 +++
\end{lstlisting}
}Лабораторная работа 4:
{
\scriptsize

\begin{lstlisting}[escapechar=!]
execve("./all", ["./all"], 0x7ffe80b64090 /* 48 vars */) = 0
mmap(NULL, 329878, PROT_READ, MAP_PRIVATE, 3, 0) = 0x7f3a9f70b000
mmap(NULL, 8192, PROT_READ|PROT_WRITE, MAP_PRIVATE|MAP_ANONYMOUS, -1, 0) = 0x7f3a9f709000
mmap(NULL, 39416, PROT_READ, MAP_PRIVATE|MAP_DENYWRITE, 3, 0) = 0x7f3a9f6ff000
mmap(0x7f3a9f701000, 16384, PROT_READ|PROT_EXEC, MAP_PRIVATE|MAP_FIXED|MAP_DENYWRITE,
 3, 0x2000) = 0x7f3a9f701000
mmap(0x7f3a9f705000, 8192, PROT_READ, MAP_PRIVATE|MAP_FIXED|MAP_DENYWRITE, 3, 0x6000) 
= 0x7f3a9f705000
mmap(0x7f3a9f707000, 8192, PROT_READ|PROT_WRITE, MAP_PRIVATE|MAP_FIXED|MAP_DENYWRITE,
 3, 0x7000) = 0x7f3a9f707000
mmap(NULL, 1848896, PROT_READ, MAP_PRIVATE|MAP_DENYWRITE, 3, 0) = 0x7f3a9f53b000
mmap(0x7f3a9f55d000, 1355776, PROT_READ|PROT_EXEC, MAP_PRIVATE|MAP_FIXED|MAP_DENYWRITE,
 3, 0x22000) = 0x7f3a9f55d000
mmap(0x7f3a9f6a8000, 311296, PROT_READ, MAP_PRIVATE|MAP_FIXED|MAP_DENYWRITE, 3, 0x16d000)
 = 0x7f3a9f6a8000
mmap(0x7f3a9f6f5000, 24576, PROT_READ|PROT_WRITE, MAP_PRIVATE|MAP_FIXED|MAP_DENYWRITE, 3,
 0x1b9000) = 0x7f3a9f6f5000
mmap(0x7f3a9f6fb000, 13888, PROT_READ|PROT_WRITE, MAP_PRIVATE|MAP_FIXED|MAP_ANONYMOUS, -1,
 0) = 0x7f3a9f6fb000
mmap(NULL, 131528, PROT_READ, MAP_PRIVATE|MAP_DENYWRITE, 3, 0) = 0x7f3a9f51a000
mmap(0x7f3a9f520000, 61440, PROT_READ|PROT_EXEC, MAP_PRIVATE|MAP_FIXED|MAP_DENYWRITE, 3,
 0x6000) = 0x7f3a9f520000
mmap(0x7f3a9f52f000, 24576, PROT_READ, MAP_PRIVATE|MAP_FIXED|MAP_DENYWRITE, 3, 0x15000) 
= 0x7f3a9f52f000
mmap(0x7f3a9f535000, 8192, PROT_READ|PROT_WRITE, MAP_PRIVATE|MAP_FIXED|MAP_DENYWRITE, 3,
 0x1a000) = 0x7f3a9f535000
mmap(0x7f3a9f537000, 12744, PROT_READ|PROT_WRITE, MAP_PRIVATE|MAP_FIXED|MAP_ANONYMOUS,
 -1, 0) = 0x7f3a9f537000
mmap(NULL, 12288, PROT_READ|PROT_WRITE, MAP_PRIVATE|MAP_ANONYMOUS, -1, 0) = 0x7f3a9f517000
mmap(NULL, 4194306, PROT_READ|PROT_WRITE, MAP_SHARED, 3, 0) = 0x7f3a9f116000
./lab01 10
10.12.63	4d4691e7cfebe79b9722e274750b4222b2470c5433e8a15dbaac06b3
0.12.83	4ea299d7ad460e7b355e60fb04b2cd99ff36b36585c6d2d6eaf87f
3.10.125	42efe7b30ef5777e4cd4b1b927f672
15.12.246	4049b4392bc0998998b375
10.10.422	80c751cea9992cd8560624ca306ebb98
11.7.475	59ce7e1d0247467518
6.0.499	1c6b85f4fce7
29.9.1096	a4c44cd3bfb6438db30d88e570bf3d87c942edcae11de5f538de12975fd385
22.7.1321	
2.8.1994	e037ae90587681c36ad18b10028d4b6359
--- SIGCHLD {si_signo=SIGCHLD, si_code=CLD_EXITED, si_pid=32458, si_uid=1000,
 si_status=0, si_utime=0, si_stime=0} ---
+++ exited with 0 +++

\end{lstlisting}
}
Лабораторная работа 5:
{
\scriptsize

\begin{lstlisting}[escapechar=!]
execve("./load", ["./load"], 0x7ffef7db4210 /* 48 vars */) = 0
brk(NULL)                               = 0x55a975dca000
arch_prctl(0x3001 /* ARCH_??? */, 0x7fffab2dc9b0) = -1 EINVAL (!Недопустимый аргумент!)
access("/etc/ld.so.preload", R_OK)      = -1 ENOENT (!Нет такого файла или каталога!)
openat(AT_FDCWD, "/etc/ld.so.cache", O_RDONLY|O_CLOEXEC) = 3
fstat(3, {st_mode=S_IFREG|0644, st_size=329878, ...}) = 0
mmap(NULL, 329878, PROT_READ, MAP_PRIVATE, 3, 0) = 0x7f0ecea6d000
close(3)                                = 0
openat(AT_FDCWD, "/usr/lib/libdl.so.2", O_RDONLY|O_CLOEXEC) = 3
read(3, "\177ELF\2\1\1\0\0\0\0\0\0\0\0\0\3\0>\0\1\0\0\0 \20\0\0\0\0\0\0"..., 832) = 832
fstat(3, {st_mode=S_IFREG|0755, st_size=14240, ...}) = 0
mmap(NULL, 8192, PROT_READ|PROT_WRITE, MAP_PRIVATE|MAP_ANONYMOUS, -1, 0) = 0x7f0ecea6b000
mmap(NULL, 16528, PROT_READ, MAP_PRIVATE|MAP_DENYWRITE, 3, 0) = 0x7f0ecea66000
mmap(0x7f0ecea67000, 4096, PROT_READ|PROT_EXEC, MAP_PRIVATE|MAP_FIXED|MAP_DENYWRITE,
 3, 0x1000) = 0x7f0ecea67000
mmap(0x7f0ecea68000, 4096, PROT_READ, MAP_PRIVATE|MAP_FIXED|MAP_DENYWRITE, 3, 0x2000) 
= 0x7f0ecea68000
mmap(0x7f0ecea69000, 8192, PROT_READ|PROT_WRITE, MAP_PRIVATE|MAP_FIXED|MAP_DENYWRITE,
 3, 0x2000) = 0x7f0ecea69000
close(3)                                = 0
openat(AT_FDCWD, "/usr/lib/libc.so.6", O_RDONLY|O_CLOEXEC) = 3
read(3, "\177ELF\2\1\1\3\0\0\0\0\0\0\0\0\3\0>\0\1\0\0\0000C\2\0\0\0\0\0"..., 832) = 832
lseek(3, 792, SEEK_SET)                 = 792
read(3, "\4\0\0\0\24\0\0\0\3\0\0\0GNU\0\201\336\t\36\251c\324\233E\371SoK\5H\334"..., 
68) = 68
fstat(3, {st_mode=S_IFREG|0755, st_size=2136840, ...}) = 0
lseek(3, 792, SEEK_SET)                 = 792
read(3, "\4\0\0\0\24\0\0\0\3\0\0\0GNU\0\201\336\t\36\251c\324\233E\371SoK\5H\334"...,
 68) = 68
lseek(3, 864, SEEK_SET)                 = 864
read(3, "\4\0\0\0\20\0\0\0\5\0\0\0GNU\0\2\0\0\300\4\0\0\0\3\0\0\0\0\0\0\0", 32) = 32
mmap(NULL, 1848896, PROT_READ, MAP_PRIVATE|MAP_DENYWRITE, 3, 0) = 0x7f0ece8a2000
mprotect(0x7f0ece8c4000, 1671168, PROT_NONE) = 0
mmap(0x7f0ece8c4000, 1355776, PROT_READ|PROT_EXEC, MAP_PRIVATE|MAP_FIXED|MAP_DENYWRITE, 
3, 0x22000) = 0x7f0ece8c4000
mmap(0x7f0ecea0f000, 311296, PROT_READ, MAP_PRIVATE|MAP_FIXED|MAP_DENYWRITE, 3, 0x16d000)
 = 0x7f0ecea0f000
mmap(0x7f0ecea5c000, 24576, PROT_READ|PROT_WRITE, MAP_PRIVATE|MAP_FIXED|MAP_DENYWRITE, 3,
 0x1b9000) = 0x7f0ecea5c000
mmap(0x7f0ecea62000, 13888, PROT_READ|PROT_WRITE, MAP_PRIVATE|MAP_FIXED|MAP_ANONYMOUS, -1,
 0) = 0x7f0ecea62000
close(3)                                = 0
mmap(NULL, 12288, PROT_READ|PROT_WRITE, MAP_PRIVATE|MAP_ANONYMOUS, -1, 0) = 0x7f0ece89f000
arch_prctl(ARCH_SET_FS, 0x7f0ece89f740) = 0
mprotect(0x7f0ecea5c000, 16384, PROT_READ) = 0
mprotect(0x7f0ecea69000, 4096, PROT_READ) = 0
mprotect(0x55a9744d2000, 4096, PROT_READ) = 0
mprotect(0x7f0eceae7000, 4096, PROT_READ) = 0
munmap(0x7f0ecea6d000, 329878)          = 0
brk(NULL)                               = 0x55a975dca000
brk(0x55a975deb000)                     = 0x55a975deb000
openat(AT_FDCWD, "./libdynamic.so", O_RDONLY|O_CLOEXEC) = 3
read(3, "\177ELF\2\1\1\0\0\0\0\0\0\0\0\0\3\0>\0\1\0\0\0p\20\0\0\0\0\0\0"..., 832) = 832
fstat(3, {st_mode=S_IFREG|0755, st_size=50360, ...}) = 0
getcwd("/home/lol/Desktop/OLD/OC/lab05/dirty", 128) = 37
mmap(NULL, 16456, PROT_READ, MAP_PRIVATE|MAP_DENYWRITE, 3, 0) = 0x7f0eceab9000
mmap(0x7f0eceaba000, 4096, PROT_READ|PROT_EXEC, MAP_PRIVATE|MAP_FIXED|MAP_DENYWRITE, 3, 
0x1000) = 0x7f0eceaba000

mmap(0x7f0eceabb000, 4096, PROT_READ, MAP_PRIVATE|MAP_FIXED|MAP_DENYWRITE, 3, 0x2000)
 = 0x7f0eceabb000
mmap(0x7f0eceabc000, 8192, PROT_READ|PROT_WRITE, MAP_PRIVATE|MAP_FIXED|MAP_DENYWRITE, 
3, 0x2000) = 0x7f0eceabc000
close(3)                                = 0
mprotect(0x7f0eceabc000, 4096, PROT_READ) = 0
fstat(1, {st_mode=S_IFCHR|0620, st_rdev=makedev(0x88, 0x1), ...}) = 0
write(1, "-1.000000 1.000000 1.000000 4.00"..., 56-1.000000 1.000000 1.000000 4.000000
 2.000000 21.992818
) = 56
munmap(0x7f0eceab9000, 16456)           = 0
exit_group(0)                           = ?
+++ exited with 0 +++
\end{lstlisting}
}
\subsection*{Выводы}

Изучение системных вызовов позволило по-настоящему понять, что делает операционная система в момент их исполнения. Важно правильно обрабатывать системные
вызовы, это поможет избежать ошибок при некорректном исполнении программы и
поможет в ее отладке. Идея такого интерфейса довольно проста: получил команду --
выполняй или отчитывайся, почему не смог. Система ЭВМ-Человек, словно ее живой аналог Человек-Человек, способна достичь высокой производительности в ходе
диалога: решения принимаются по ходу дела, а не планируются за сотню ходов. Если
что-то пошло не так, при правильной обработке ошибок мы сразу получим сообщение о ней в этом месте программы.

Утилиты диагностики позволяют быстро выяснить, как программа взаимодействует
с операционной системой. Это происходит путем мониторинга системных вызовов
и сигналов. Становится понятнее работа собственного и стороннего программного
обеспечения. Можно узнать, что делают конкретные системные вызовы, подсчитать
их, узнать время их работы и ошибки при их использовании, если они есть, отследить запущенные процессы.
В случае, если у нас нет доступа к исходному коду, мы можем воспользоваться этими утилитами и узнать, что действительно происходит в программе. Отчаявшийся
пользователь может использовать утилиты диагностики для отладки своего кода.
Они позволяют увидеть не только, где программа ``упала'' но и почему это произошло. Для эффективного использования трассировки необходимо знать системные
вызовы и понимать их работу. Использование такого инструмента как strace жизненно необходимо любому уважающему себя программисту.

\end{document}

