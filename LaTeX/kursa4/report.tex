\documentclass[12pt]{article}

\usepackage[utf8x]{inputenc}
\usepackage[T1, T2A]{fontenc}
\usepackage{fullpage}
\usepackage{multicol,multirow}
\usepackage{tabularx}
\usepackage{ulem}
\usepackage{listings} 
\usepackage[english,russian]{babel}
\usepackage{tikz}
\usepackage{pgfplots}
\usepackage{indentfirst}
\usepackage{ulem} 


\parindent=1cm
\makeatletter
\newcommand{\rindex}[2][\imki@jobname]{%
    \index[#1]{\detokenize{#2}}%
}
\makeatother
\newcolumntype{P}[1]{>{\raggedbottom\arraybackslash}p{#1}}

\linespread{1}
\pgfplotsset{compat=1.16}
\begin{document}

\section*{\centering Курсовой проект по курсу:\\ Операционные системы}

\noindent Выполнил студент группы М8О-208Б-17 МАИ \,\, \textit{Милько Павел}.

\subsection*{Цель работы}

\noindent Необходимо спроектировать и реализовать программный прототип в соответствии с
выбранным вариантом. Произвести анализ и сделать вывод на основании данных,
полученных при работе программного прототипа.


\subsection*{Задание}
%\textbf{Вариант 13.}

Проектирование консольной клиент-серверной игры точки и квадраты. Должна быть поддержка нескольких игр сразу. Сделать 2 режима игры - с полем 3х3 и 9х9. При отключении одного из игроков второй автоматически выигрывает. В качестве технологии обмена данными использовать сервер сообщений ZeroMQ.

\subsection*{Информация}

Для реализации связи клиент-сервер было использовано несколько паттернов:  RequestResponse для инициализации клиента на сервере, и для отправки ходов, отдельный сокет для проверки соединения клиента с сервером, сокет типа SubscriberPublisher для одновременной отправки текущего состояния игры клиентам.

Сервер запускается с ключом количества одновременных игр (дефолтное значение 5). При запуске сервер запрашивает номер порта, где будет запущен сокет подключения. При подключении клиента к серверу выделяется разделяемая память под игровую сессию. Если клиент подключился к серверу и уже есть одна свободная игра с выбранным режимом, то клиент будет подключён к ней. Обеспечение игры производится  с помощью функции, работающей в отдельном потоке (таким образом игры на самом деле происходят одновременно). 

Первый подключившийся к игре клиент получает идентификатор А, второй клиент - В. Если один из клиентов отключится от игры по техническим причинам, то второй автоматически станет победителем.

{\bf Системные вызовы:}
\begin{enumerate}
    \item \lstinline[language=c]|void *zmq_ctx_new ();| -- создаёт новый контекст, возвращает его адрес, в случае неудачи вернётся $NULL$
    \item \lstinline[language=c]|void *zmq_socket(void *context, int type);| -- создает сокет типа type из котнекста context.
    \item \lstinline[language=c]|int zmq_connect(void *socket, const char *endpoint);| -- подключает socket к пути endpoint, 0 в случае успеха, -1 в случае ошибки.
    \item \lstinline[language=c]|int zmq_bind(void *socket, const char *endpoint);| -- присоединяет socket к пути endpoint, 0 в случае успеха, -1 в случае ошибки.
    \item \lstinline[language=c]|int zmq_msg_init(zmq_msg_t *msg);| -- инициализирует сообщение msg как пустой объект.
    \item \lstinline[language=c]|int zmq_msg_init_size (zmq_msg_t *msg, size_t size);| -- инициализирует сообщение msg для отправки, устанавливает размер сообщения в size байт. При успехе возвращет 0, при неудаче -1
    \item \lstinline[language=c]|int zmq_msg_send(zmq_msg_t *msg, void *socket, int flags);| -- отправляет сообщение msg в socket с параметрами flags, возвращает количество отправленных байт, в случае ошибки возвращает -1.
    \item \lstinline[language=c]|int zmq_msg_recv(zmq_msg_t *msg, void *socket, int flags);| -- получает сообщение из socket в msg с параметрами flags, возвращает количество полученных байт, в случае ошибки возвращает -1.
    \item \lstinline[language=c]|int zmq_msg_close(zmq_msg_t *msg);| -- очищает содержимое msg, аналог free для сообщений zmq, возвращает 0 в случае успеха и -1 в случае неудачи.
    \item \lstinline[language=c]|int zmq_close(void *socket);| -- закрывает сокет socket, возвращает 0 в случае успеха и -1 в случае неудачи.
    \item \lstinline[language=c]|int zmq_ctx_destroy(void *context);| -- разрушает контекст context, блокирует доступ всем операциям кроме \lstinline|zmq_close|, все сообщения в сокетах либо физически отправлены, либо "висят".
\end{enumerate}



\subsection*{Описание программы}

Игровые сессии хранятся на сервере в виде массива, структура одной игровой сессии выглядит следующим образом:
{\footnotesize \begin{lstlisting}[language=C]
typedef struct game {
    int A, B, pub_id;
    void *pub, *A_sub, *B_sub;
    field_t* field;
    uint num;
    point_t point;
} game_t;
\end{lstlisting}}

Где A,B это идентификационные номера игроков, которые они указали при подключении к серверу. pub\_id - номер порта, на котором будет запущен сокет Subscriber-Publisher. Собственно, сам сокет - pub. A\_sub и B\_sub - сокеты для проверки соединения клиентов. field - игровое поле. num - номер игровой сессии. point - энумератор, обозначающий какой игрок сейчас ходит.
\newline


Основные структуры и функции определены в файле {\bf game.h}:
{\footnotesize \begin{lstlisting}[language=C]
#ifndef GAME_H
#define GAME_H

#include <pthread.h>
#include <stdio.h>
#include <stdlib.h>
#include <string.h>
#include <sys/types.h>
#include <zmq.h>
#include <stdint.h>

typedef uint8_t uint;
#define WRONG_TURN -1

typedef enum {
    A,
    B,
    nothing
} point_t;

typedef enum side {
    top,
    right,
    bottom,
    left
} side_t;

typedef struct cell {
    unsigned char edges[4];
    point_t type;
} cell_t;

typedef struct turn {
    uint x, y;
    side_t side;
} turn_t;

typedef struct field {
    cell_t** regiment;
    uint size;
    char end;
} field_t;

typedef struct Message {
    turn_t turn;
    int id, req;
    point_t point;
    uint size;
} Message_t;

field_t gen_field(uint size);
void destroy_field(field_t F);

void show_field(field_t f);
int fix_field(point_t player, field_t* f);

void get_turn(uint x, uint y, side_t side, field_t f);
int try_turn(field_t* f, turn_t choice, point_t player);

void send_field(field_t field, void* requester);
void get_field(field_t* field, void* resiever);
#endif // GAME_H
\end{lstlisting}}

\subsection*{Исходный код}

{\footnotesize
\lstinputlisting[escapechar=~,language=C]{../../kursa4/server.c}
	
\lstinputlisting[escapechar=~,language=C]{../../kursa4/client.c}
}

\subsection*{Тестирование}

К серверу по очереди подключаются 4 игрока - 2 в режиме 3х3 и 2 в режиме 9х9:
{\footnotesize \begin{lstlisting}
Server started, max count of games is: 5
Enter server's adress: 1234
Server started
Created new game, size=3, adress=tcp://*:2001,game_id=0
Players joined to game 0: player_A=1, player_B=2
Game 0 deleted
Created new game, size=9, adress=tcp://*:2001,game_id=0
Players joined to game 0: player_A=3, player_B=4
Game 0 deleted
\end{lstlisting}}
Так как клиенты подключались по очереди, игры происходили последовательно. 
\newline

\noindent К серверу сначала подключается игрок с режимом 3х3, затем 9х9  и снова 3х3 и 9х9:
{\footnotesize \begin{lstlisting}
Server started, max count of games is: 5
Enter server's adress: 1234
Server started
Created new game, size=3, adress=tcp://*:2001,game_id=0
Created new game, size=9, adress=tcp://*:2002,game_id=1
Players joined to game 0: player_A=1, player_B=2
Game 0 deleted
Players joined to game 1: player_A=3, player_B=4
Game 1 deleted
\end{lstlisting}}

\noindent К серверу подключается игрок с режимом 3х3, затем игрок 9х9. Потом подключается, но не играет ещё один игрок с режимом 3х3, затем подключается игрок с режимом 9х9 и играет сессию до конца. Продолжают играть два игрока с режимом 3х3.

{\footnotesize \begin{lstlisting}
Server started, max count of games is: 5
Enter server's adress: 1234
Server started
Created new game, size=3, adress=tcp://*:2001,game_id=0
Created new game, size=9, adress=tcp://*:2002,game_id=1
Players joined to game 0: player_A=1, player_B=2
Players joined to game 1: player_A=3, player_B=4
Game 0 deleted
Game 1 deleted
\end{lstlisting}}


\subsection*{Выводы}

В данном курсовом проекте было использовано практически всё, что содержится в курсе операционных систем, начиная от мьютексов и работы с потоками, и заканчивая сокетами и разделяемой памятью. 

Я использовал эти технологии в курсовом проекте потому что это сильно упрощает разработку программ и многократно увеличивает возможности для расширения функционала готового приложения. Например, можно несколько изменить код сервера, и тогда он сможет запоминать статистику игр и восстанавливать её, если программа неожиданно завершится или возникнет техническая неисправность. 

Путём добавления пары строчек можно реализовать обмен небольшими сообщениями между клиентами, или сделать клиента-администратора, который сможет управлять сервером.

Курс операционных систем был очень интересным  и познавательным. теперь я больше знаю о возможностях операционной системы и специфических средствах отладки, в неё встроенных. Хотя лабораторные работы были небольшими и не очень сложными, они отнимали много времени на осмысление того что от меня надо и как это сделать. 

\end{document}

