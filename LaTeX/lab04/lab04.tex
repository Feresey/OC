\documentclass[12pt]{article}

\usepackage[utf8x]{inputenc}
\usepackage[T1, T2A]{fontenc}
\usepackage{fullpage}
\usepackage{multicol,multirow}
\usepackage{tabularx}
\usepackage{ulem}
\usepackage{listings} 
\usepackage[english,russian]{babel}
\usepackage{tikz}
\usepackage{pgfplots}
\usepackage{indentfirst}
\usepackage{ulem} 


\parindent=1cm
\makeatletter
\newcommand{\rindex}[2][\imki@jobname]{%
    \index[#1]{\detokenize{#2}}%
}
\makeatother
\newcolumntype{P}[1]{>{\raggedbottom\arraybackslash}p{#1}}

\linespread{1}
\pgfplotsset{compat=1.16}
\begin{document}

\section*{\centering Лабораторная работа №\,4 по курсу :\\ Операционные системы}

\noindent Выполнил студент группы М8О-208Б-17 МАИ \,\, \textit{Милько Павел}.

\subsection*{Цель работы}

\noindent Приобретение практических навыков в:
\begin{itemize}
\item  Освоение принципов работы с файловыми системами
\item Обеспечение обмена данных между процессами посредством технологии <<File 

 mapping>>
    
\end{itemize}

Составить и отладить программу на языке Си, осуществляющую работу с процессами и взаимодействие между ними в одной из двух операционных систем. В результате работы программа (основной процесс) должен создать для решение задачи один или несколько дочерних процессов. Взаимодействие между процессами осуществляется через системные сигналы/события и/или через отображаемые файлы (memory-mapped files).
Необходимо обрабатывать системные ошибки, которые могут возникнуть в результате работы.


\subsection*{Задание}
\textbf{Вариант 16.}

На вход программе подается команда интерпретатора команд и имя файла. Программа должна перенаправить стандартный ввод команды с этого файла и вывести результат команды в стандартный выходной поток. Использование операций write и printf запрещено.

\subsection*{Информация}

Собственно, цель программы вызвать с помощью exec* некую программу и перенаправить ввод с указанного файла.

Отличие от второй лабораторной лишь в том, что нужно использовать mmap для отображения виртуальной памяти. 
Самое сложное при выполнении лабораторной было как раз придумать куда этот mmap впихнуть.

Мною было принято решение использовать эту возможность для ввода имени команды и файла.

Логично создавать для вызова exec отдельный процесс, так как эта функция является системным вызовом и замещает собой породивший её процесс.
\\
Для переопределения дескриптора использовалась функция:

\lstinline[language=c]|FILE *freopen(const char *pathname, const char *mode, FILE *stream);|

\noindent Вызов команды интерпретатора осуществляется функцией:

% \\    \smallbreak 
\lstinline[language=c]|execlp(command, command, NULL);|

Где $command$ --- это имя команды интерпретатора.

\subsection*{Описание программы}

{\scriptsize
\begin{lstlisting}[language=c,escapechar=@]
#include <ctype.h>
#include <errno.h>
#include <fcntl.h>
#include <malloc.h>
#include <semaphore.h>
#include <signal.h>
#include <stdio.h>
#include <stdlib.h>
#include <string.h>
#include <sys/mman.h>
#include <sys/stat.h>
#include <sys/types.h>
#include <sys/wait.h>
#include <unistd.h>

const int ARG_MAX = 2097152 + 1;

int main()
{
    pid_t pid;
    int rv, fd;
    char* memory;
    // @создаем разделяемую память@
    if ((fd = shm_open("shared_file", 
       O_RDWR | O_CREAT | O_TRUNC, 
       S_IRUSR | S_IWUSR)) == -1) {
        perror("shm::open_fail");
        exit(-1);
    }
    
    //@задаем ей нужный размер@
    if (ftruncate(fd, ARG_MAX * 2) == -1) {
        perror("trucate::fail");
        exit(-1);
    }
    //@отображаем на вирт. память@
    memory = mmap(NULL, ARG_MAX * 2, PROT_READ | PROT_WRITE, MAP_SHARED, fd, 0); 
    
    if (memory == MAP_FAILED) {
        perror("mmap::mapping_fail");
        fprintf(stderr, "%p", memory);
        exit(-1);
    }
    close(fd);
    
    char* command = memory;
    char* filename = memory + ARG_MAX;
    
    *command = *filename = '\0';
    
    switch (pid = fork()) {
        case -1: {
            perror("fork");
            exit(1);
        }
        
        case 0: {
            //@Потомок@
            while (*command == '\0' || *filename == '\0')
            sleep(1);
            FILE* my = freopen(filename, "r", stdin);
            if (!my) {
                perror("File ERROR");
                exit(errno);
            }
            rv = execlp(command, command, NULL);
            if (rv)
                perror("Exec ERROR");
            fclose(my);
            _exit(rv);
        }
        
        default: {
            //@Родитель@
            scanf("%s %s", command, filename);
            waitpid(pid, &rv, 0);
            exit(WEXITSTATUS(rv));
        }
    }
    if (munmap(memory, ARG_MAX * 2)) {
        perror("munmap");
        exit(-1);
    }
    
    munmap(memory, ARG_MAX * 2);
    return 0;
}
\end{lstlisting}
}

\noindent Системные вызовы:
\begin{enumerate}
    \item \lstinline[language=c]|int shm_open(const char *name, int oflag, mode_t mode);|  -- создаёт новый разделяемый файл, который может предоставить доступ на чтение и запись сразу нескольким процессам. При успехе возвращает файловый дескриптор на соответствующий файл, при неудаче -1.
     
    Были использованы следующие флаги:
    \subitem O\_RDWR    Open the object for read-write access.
    \subitem O\_CREAT    Create  the shared memory object if it does not exist.
    \subitem O\_TRUNC    If the shared memory object already exists, truncate it to zero bytes.
    
    
    \item  \lstinline[language=c]|int ftruncate(int fildes, off_t length);| -- устанавливает размер файла в length байт. При успехе возвращает 0. Если указан неверный дескриптор, или дескриптор указывает на директорию, или возникла другая ошибка возвращается -1.
    \item \lstinline[language=c]|void *mmap(void *addr, size_t len, int prot, int flags, int fildes, off_t off);| 
    
    -- устанавливает отображение между адресным пространством процесса и объектом памяти.
         
    Были использованы следующие флаги:
    \subitem PROT\_READ   Data can be read.
    \subitem PROT\_WRITE   Data can be written. 
    \subitem MAP\_SHARED    Changes are shared. 
    \item \lstinline[language=c]|int munmap(void *addr, size_t len);|  -- удаляет все сопоставления для всех страниц, содержащих любую часть адресного пространства процесса, начиная с addr и на len байт.
\end{enumerate}

\subsection*{Вывод strace}
{\begin{lstlisting}
openat(AT_FDCWD, "/dev/shm/shared_file",
 O_RDWR|O_CREAT|O_TRUNC|O_NOFOLLOW|O_CLOEXEC, 0600) = 3
ftruncate(3, 4194306)                   = 0
mmap(NULL, 2097153, 
PROT_READ|PROT_WRITE, MAP_SHARED, 3, 0) = 0x7f495335c000
\end{lstlisting}}

\subsection*{Дневник отладки}

Ошибок не возникало, да и сложно их сделать при таком простом задании. Намного больше времени отняло чтение мануалов по этим командам.

\subsection*{Выводы}

Были приобретены минимальные навыки для работы с разделяемой памятью. К сожалению их использование в данной лабораторной непрактично, но иногда бывает удобно общаться между потоками при помощи shared файлов.

\end{document}

