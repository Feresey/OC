\documentclass[12pt]{article}

\usepackage[utf8x]{inputenc}
\usepackage[T1, T2A]{fontenc}
\usepackage{fullpage}
\usepackage{multicol,multirow}
\usepackage{tabularx}
\usepackage{ulem}
\usepackage{listings} 
\usepackage[english,russian]{babel}
\usepackage{tikz}
\usepackage{pgfplots}
\usepackage{indentfirst}
\usepackage{ulem} 


\parindent=1cm
\makeatletter
\newcommand{\rindex}[2][\imki@jobname]{%
    \index[#1]{\detokenize{#2}}%
}
\makeatother
\newcolumntype{P}[1]{>{\raggedbottom\arraybackslash}p{#1}}

\linespread{1}
\pgfplotsset{compat=1.16}
\begin{document}

\section*{\centering Лабораторная работа №\,6 по курсу :\\ Операционные системы}

\noindent Выполнил студент группы М8О-208Б-17 МАИ \,\, \textit{Милько Павел}.

\subsection*{Цель работы}

\noindent Реализовать клиент-среверную систему по асинхронной обработке запросов. Необходимо составить программы сервера и клиента. При запуске сервер и клиент должны быть настраиваемы, то есть должна быть возможность поднятия на одной ЭВМ нескольких серверов по обработке данных и нескольких клиентов, которые к ним относятся. Все общение между процессами сервера и клиентов должно осуществляться
через сервер сообщений. Серверное приложение -- банк. Клиентское приложение -- клиент банка. Клиент может отправить какую-то денежную сумму в банк на хранения. Клиент также может запросить из банка произвольную сумму. Клиенты могут посылать суммы на счета других клиентов. Запросить собственный счет. При снятии должна производиться проверка на то, что у клиента достаточно денег для снятия денежных средств. Идентификатор клиента задается во время запуска клиентского приложения, как и адрес банка. Считать, что идентификаторы при запуске клиентов будут уникальными.



\subsection*{Задание}
\textbf{Вариант 13.}

\begin{itemize}
    \item Сервер сообщений: ZeroMQ.
    \item Внутреннее хранилище сервера: линейный список (однонаправленный).
    \item Тип ключа клиента: вещественный 64-битный.
    \item Дополнительные возможности сервера: возможность временной приостановки работы сервера без выключения. Сообщения серверу можно отправлять, но ответы сервер не отправляет до возобновления работы.
\end{itemize}

\subsection*{Информация}

Для реализации связи клиент-сервер был выбран паттерн Request-Response. Клиент отправляет запрос на сервер и ждёт ответа. После того, как ответ пришел, клиент может продолжать работу. Клиент подключается к серверу, производит проверку на наличие клиента в базе, затем работает с сервером. Сервер обрабатывает запросы
клиента, смотрит его наличие и состояние баланса, в случае ошибки или недостатке средств посылает соответствующий ответ клиенту, также выводит лог действий на stdout. Если сервер заблокирован, то он не принимает сообщения от клиентов пока его не разблокируют.


\noindent Как только работа сервера возобновлена, он продолжает обрабатывать новые приходящие запросы.

Системные вызовы:
\begin{enumerate}
    \item \lstinline[language=c]|void *zmq_ctx_new ();| -- создаёт новый контекст, возвращает его адрес, в случае неудачи вернётся $NULL$
    \item \lstinline[language=c]|void *zmq_socket(void *context, int type);| -- создает сокет типа type из котнекста context.
    \item \lstinline[language=c]|int zmq_connect(void *socket, const char *endpoint);| -- подключает socket к пути endpoint, 0 в случае успеха, -1 в случае ошибки.
    \item \lstinline[language=c]|int zmq_bind(void *socket, const char *endpoint);| -- присоединяет socket к пути endpoint, 0 в случае успеха, -1 в случае ошибки.
    \item \lstinline[language=c]|int zmq_msg_init(zmq_msg_t *msg);| -- инициализирует сообщение msg как пустой объект.
    \item \lstinline[language=c]|int zmq_msg_init_size (zmq_msg_t *msg, size_t size);| -- инициализирует сообщение msg для отправки, устанавливает размер сообщения в size байт. При успехе возвращет 0, при неудаче -1
    \item \lstinline[language=c]|int zmq_msg_send(zmq_msg_t *msg, void *socket, int flags);| -- отправляет сообщение msg в socket с параметрами flags, возвращает количество отправленных байт, в случае ошибки возвращает -1.
    \item \lstinline[language=c]|int zmq_msg_recv(zmq_msg_t *msg, void *socket, int flags);| -- получает сообщение из socket в msg с параметрами flags, возвращает количество полученных байт, в случае ошибки возвращает -1.
    \item \lstinline[language=c]|int zmq_msg_close(zmq_msg_t *msg);| -- очищает содержимое msg, аналог free для сообщений zmq, возвращает 0 в случае успеха и -1 в случае неудачи.
    \item \lstinline[language=c]|int zmq_close(void *socket);| -- закрывает сокет socket, возвращает 0 в случае успеха и -1 в случае неудачи.
    \item \lstinline[language=c]|int zmq_ctx_destroy(void *context);| -- разрушает контекст context, блокирует доступ всем операциям кроме \lstinline|zmq_close|, все сообщения в сокетах либо физически отправлены, либо "висят".
\end{enumerate}

\newpage

\subsection*{Описание программы}

\noindent Для простоты обмена сообщениями данные передавались только с помощью одной структуры:

{ \scriptsize
\begin{lstlisting}[language=c]
#define STR_SIZE 256
#define ID double

typedef struct _msg {
    ID    client;
    int   sum;
    int   action;
    ID    receiverClient;
    void* requester;
    char  message[STR_SIZE];
} Message_t;

\end{lstlisting}
}

\noindent Для реализации функций банка были использованы следующие структуры и функции: 
{ \scriptsize
\begin{lstlisting}[language=c]
#include <inttypes.h>
#include <stdint.h>
#include <stdio.h>
#include <stdlib.h>

#include "message.h"

#define SUCCESS 1
#define NOT_MEMORY -1
#define NOT_ENOUGH_MONEY -2
#define NOT_CLIENT -3
#define RECEIVER_NOT_CLIENT -4

typedef struct _client {
    ID id;
    int sum;
} Client;

typedef struct _client_base {
    Client client;
    struct _client_base* next;
} ClientBase;

ClientBase* Bank_init();
void Bank_add_client(ClientBase*, ID client);
void Bank_print_client_base(ClientBase*);
Client* _Bank_find_client(ClientBase*, double clientID);
void Bank_destroy_database(ClientBase*);
int Bank_destroy_account(ClientBase*, double clientID);

void Bank_put_money(double ClientID, int sum, ClientBase*);
int Bank_get_money(double ClientID, int sum, ClientBase*);
int Bank_send_money(ID clientSender, ID clientReceiver, int sum, ClientBase*);
int Bank_get_info(ClientBase*, ID client);
\end{lstlisting}
}
\newpage
\subsection*{Дневник отладки}

При написании лабораторной возникали проблемы с передачей сообщений (отправлял не на сокет клиента, а на свой собственный, естественно мне никто не отвечал). И с утечками памяти (забыл сделать zmq\_msg\_close).

 Также из за неверно введённых данных циклически отправлялось сообщение последнему клиенту (забыл пункт default внутри switch-case в основном цикле)

\subsection*{Выводы}

Это была интересная лабораторная, потому что у нас была относительная свобода в выборе архитектурных решений. Это всегда лучше, чем следовать заранее определённому плану (остаётся больше возможностей для костылей). Благодаря работе с сервером сообщений, я узнал о сокетах, и различных видах паттернов передачи сообщений.

При построении серверов, которые сохраняют состояние клиентов, возникают классические проблемы. Если сервер хранит состояние каждого клиента, а клиентов
подключается все больше и больше, в конце концов наступает момент исчерпания
ресурсов. Даже если одни и те же клиенты сохраняют коннект и используется идентификация по умолчанию, то каждое соединение будет выглядеть как новое.

Код из данной лабораторной лучше никому не показывать во избежания несчастных случаев. О безопасности подключения клиентов лучше не вспоминать, удобство интерфейса на уровне плинтуса. Единственный плюс моего кода -- стабильность на больших данных и отказоустойчивость при неверных запросах.

Работа с сокетами была интересной и познавательной, а так же непривычной (геморройной). Мне кажется что я собрал все возможные ошибки, начиная с сигфолта сервера при получении сообщения от клиента и заканчивая fork-бомбой, из за которой пришлось перезагружать компьютер 4 раза.

\end{document}

