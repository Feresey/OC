\documentclass[12pt]{article}

\usepackage[utf8x]{inputenc}
\usepackage[T1, T2A]{fontenc}
\usepackage{fullpage}
\usepackage{multicol,multirow}
\usepackage{tabularx}
\usepackage{ulem}
\usepackage{listings} 
\usepackage[english,russian]{babel}
\usepackage{tikz}
\usepackage{pgfplots}
\usepackage{indentfirst}
\usepackage{ulem} 


\parindent=1cm
\makeatletter
\newcommand{\rindex}[2][\imki@jobname]{%
    \index[#1]{\detokenize{#2}}%
}
\makeatother
\newcolumntype{P}[1]{>{\raggedbottom\arraybackslash}p{#1}}

\hfuzz=10000pt
\vbadness10000
\linespread{1}
\pgfplotsset{compat=1.16}
\begin{document}

\section*{\centering Лабораторная работа №\,1 по курсу :\\ Операционные системы}

Выполнил студент группы М8О-208Б-17 МАИ \,\, \textit{Милько Павел}.

\subsection*{Условие}

%Описание задачи: 
\begin{itemize}
    \item Постановка задачи:\subitem Составить и отладить программу на языке Си, осуществляющую работу с процессами и взаимодействие между ними в операционной системе Linux. В результате работы программа (основной процесс) должен создать для решение задачи один или несколько дочерних процессов. Взаимодействие между процессами осуществляется через системные сигналы/события и/или каналы (pipe).
    Необходимо обрабатывать системные ошибки, которые могут возникнуть в результате работы. 
    
    \item Вариант задания  \subitem На вход программе подаётся команда интерпретатора команд и имя файла. Программа должна перенаправить стандартный ввод команды с этого файла и вывести результат команды в стандартный выходной поток. Использование операций $write$ и $printf$ запрещено.
\end{itemize}

\subsection*{Метод решения}

Суть программы заключается в переопределении файлового дескриптора для stdio и вызова команды с помощью exec.

Логично создавать для вызова exec отдельный процесс, так как эта функция является системным вызовом и замещает собой породивший её процесс.
\\
Для переопределения дескриптора использовалась функция:

\lstinline freopen(file, "r", stdin);

Где $file$ --- это имя файла, подающееся на вход вызываемой команде.\\
Вызов команды интерпретатора осуществляется функцией:
% \\    \smallbreak 
\lstinline execlp(command, command, NULL);


Где $command$ --- это имя команды интерпретатора.

\subsection*{Описание программы}

\begin{lstlisting}[escapechar=|]
#include <stdio.h>
#include <stdlib.h>
#include <unistd.h>
#include <sys/types.h>
#include <sys/wait.h>
#include <errno.h>

int main()
{
    pid_t pid;
    int rv;
    char command[100];
    char file[100];
    
    switch (pid = fork()){
        case -1:
            perror("fork"); // |произошла ошибка|
            exit(1);        //| выход из родительского процесса|
        case 0: //| Потомок|
            scanf("%s\n%s", command, file);
            freopen(file, "r", stdin);
            if (errno){
                perror("File ERROR");
                exit(errno);
    }
    
    rv = execlp(command, command, NULL);
    if (rv)
    perror("Exec ERROR");
    _exit(rv);
    default: |// Родитель|
    waitpid(pid, &rv, 0);
    exit(WEXITSTATUS(rv));
    }
}
    
\end{lstlisting}
%\newpage
%\lstinputlisting{../../lab0/main.c}
%\includeonly{"../../lab02/main.c"}
%\include{../../lab02/main.c}

\subsection*{Дневник отладки}  

Для начала я освежил \sout{отсутствующие} знания по созданию процессов и связи между ними.

Затем производил эксперименты с функцией $exec$ и её производными.
В процессе изучения я понял, что путь для программы необходимо указывать относительно (в случае если команда не содержится в $/usr/bin$), и относительным $./path/to/file$ если она не лежит в $/usr/bin$.

\subsection*{Недочёты}

Задача данной лабораторной заключается именно в получении базовых знаний работы с процессами и системными вызовами. Практической необходимости создания нескольких процессов по моему варианту задания не требуется (за меня всё сделает $exec$)

\subsection*{Выводы}

Вызов $fork$ дублирует породивший его процесс со всеми его переменными, файловыми дескрипторами, приоритетами процесса, рабочий и корневой каталоги, и сегментами выделенной памяти.

Ребёнок {\bf не} наследует:
\begin{itemize}
    \item идентификатора процесса (PID, PPID);
    \item израсходованного времени ЦП (оно обнуляется);
    \item сигналов процесса-родителя, требующих ответа;
    \item блокированных файлов (record locking).
\end{itemize}
\end{document}

