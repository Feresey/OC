\documentclass[12pt]{article}

\usepackage[utf8x]{inputenc}
\usepackage[T1, T2A]{fontenc}
\usepackage{fullpage}
\usepackage{multicol,multirow}
\usepackage{tabularx}
\usepackage{ulem}
\usepackage{listings} 
\usepackage[english,russian]{babel}
\usepackage{tikz}
\usepackage{pgfplots}
\usepackage{indentfirst}
\usepackage{ulem} 


\parindent=1cm
\makeatletter
\newcommand{\rindex}[2][\imki@jobname]{%
    \index[#1]{\detokenize{#2}}%
}
\makeatother
\newcolumntype{P}[1]{>{\raggedbottom\arraybackslash}p{#1}}

\linespread{1}
\pgfplotsset{compat=1.16}
\begin{document}

\section*{\centering Лабораторная работа №\,2 по курсу :\\ Операционные системы}

Выполнил студент группы М8О-208Б-17 МАИ \,\, \textit{Милько Павел}.

\subsection*{Условие}

%Описание задачи: 
\begin{itemize}
    \item Постановка задачи:\subitem Составить и отладить программу на языке Си, осуществляющую работу с процессами и взаимодействие между ними в операционной системе Linux. В результате работы программа (основной процесс) должен создать для решение задачи один или несколько дочерних процессов. Взаимодействие между процессами осуществляется через системные сигналы/события и/или каналы (pipe).
    Необходимо обрабатывать системные ошибки, которые могут возникнуть в результате работы. 
    
    \item Вариант задания  \subitem На вход программе подаётся команда интерпретатора команд и имя файла. Программа должна перенаправить стандартный ввод команды с этого файла и вывести результат команды в стандартный выходной поток. Использование операций $write$ и $printf$ запрещено.
\end{itemize}

\subsection*{Метод решения}

Суть программы заключается в переопределении файлового дескриптора для stdio и вызова команды с помощью exec.

Логично создавать для вызова exec отдельный процесс, так как эта функция является системным вызовом и замещает собой породивший её процесс.
\\
Для переопределения дескриптора использовалась функция:

\lstinline| freopen(file, "r", stdin);|

Где $file$ --- это имя файла, подающееся на вход вызываемой команде.\\
Вызов команды интерпретатора осуществляется функцией:

% \\    \smallbreak 
\lstinline| execlp(command, command, NULL);|


Где $command$ --- это имя команды интерпретатора.

\subsection*{Описание программы}

\begin{lstlisting}[escapechar=|]
#include <stdio.h>
#include <stdlib.h>
#include <unistd.h>
#include <sys/types.h>
#include <sys/wait.h>
#include <errno.h>

int main()
{
    pid_t pid;
    int rv;
    char command[100];
    char file[100];
    
    switch (pid = fork()){
        case -1:
            perror("fork"); // |произошла ошибка|
            exit(1);        //| выход из родительского процесса|
        case 0: //| Потомок|
            scanf("%s\n%s", command, file);
            freopen(file, "r", stdin);
            if (errno){
                perror("File ERROR");
                exit(errno);
    }
    
    rv = execlp(command, command, NULL);
    if (rv)
    perror("Exec ERROR");
    _exit(rv);
    default: |// Родитель|
    waitpid(pid, &rv, 0);
    exit(WEXITSTATUS(rv));
    }
}
    
\end{lstlisting}
%\newpage
%\lstinputlisting{../../lab0/main.c}
%\includeonly{"../../lab02/main.c"}
%\include{../../lab02/main.c}

\subsection*{Вывод  strace}
{\scriptsize 
\begin{lstlisting}[escapechar=!]
execve("./main", ["./main"], 0x7ffcf85d5850 /* 75 vars */) = 0
brk(NULL)                               = 0x557ef8389000
arch_prctl(0x3001 /* ARCH_??? */, 0x7fff8a829f50) = -1 EINVAL (!Недопустимый аргумент!)
access("/etc/ld.so.preload", R_OK)      = -1 ENOENT (!Нет такого файла или каталога!)
openat(AT_FDCWD, "/etc/ld.so.cache", O_RDONLY|O_CLOEXEC) = 3
fstat(3, {st_mode=S_IFREG|0644, st_size=362838, ...}) = 0
mmap(NULL, 362838, PROT_READ, MAP_PRIVATE, 3, 0) = 0x7f34cda1f000
close(3)                                = 0
openat(AT_FDCWD, "/usr/lib/libpthread.so.0", O_RDONLY|O_CLOEXEC) = 3
read(3, "\177ELF\2\1\1\3\0\0\0\0\0\0\0\0\3\0>\0\1\0\0\0\340f\0\0\0\0\0\0"..., 832) = 832
fstat(3, {st_mode=S_IFREG|0755, st_size=155408, ...}) = 0
mmap(NULL, 8192, PROT_READ|PROT_WRITE, MAP_PRIVATE|MAP_ANONYMOUS, -1, 0) = 0x7f34cda1d000
lseek(3, 808, SEEK_SET)                 = 808
read(3, "\4\0\0\0\20\0\0\0\5\0\0\0GNU\0\2\0\0\300\4\0\0\0\3\0\0\0\0\0\0\0", 32) = 32
mmap(NULL, 131528, PROT_READ, MAP_PRIVATE|MAP_DENYWRITE, 3, 0) = 0x7f34cd9fc000
mmap(0x7f34cda02000, 61440, PROT_READ|PROT_EXEC, MAP_PRIVATE|MAP_FIXED|MAP_DENYWRITE, 3,
 0x6000) = 0x7f34cda02000
mmap(0x7f34cda11000, 24576, PROT_READ, MAP_PRIVATE|MAP_FIXED|MAP_DENYWRITE, 3, 0x15000) 
= 0x7f34cda11000
mmap(0x7f34cda17000, 8192, PROT_READ|PROT_WRITE, MAP_PRIVATE|MAP_FIXED|MAP_DENYWRITE, 3,
 0x1a000) = 0x7f34cda17000
mmap(0x7f34cda19000, 12744, PROT_READ|PROT_WRITE, MAP_PRIVATE|MAP_FIXED|MAP_ANONYMOUS, 
-1, 0) = 0x7f34cda19000
close(3)                                = 0
openat(AT_FDCWD, "/usr/lib/libm.so.6", O_RDONLY|O_CLOEXEC) = 3
read(3, "\177ELF\2\1\1\3\0\0\0\0\0\0\0\0\3\0>\0\1\0\0\0 \320\0\0\0\0\0\0"..., 832) = 832
fstat(3, {st_mode=S_IFREG|0755, st_size=1587104, ...}) = 0
mmap(NULL, 1589272, PROT_READ, MAP_PRIVATE|MAP_DENYWRITE, 3, 0) = 0x7f34cd877000
mmap(0x7f34cd884000, 659456, PROT_READ|PROT_EXEC, MAP_PRIVATE|MAP_FIXED|MAP_DENYWRITE, 
3, 0xd000) = 0x7f34cd884000
mmap(0x7f34cd925000, 872448, PROT_READ, MAP_PRIVATE|MAP_FIXED|MAP_DENYWRITE, 3, 0xae000) 
= 0x7f34cd925000
mmap(0x7f34cd9fa000, 8192, PROT_READ|PROT_WRITE, MAP_PRIVATE|MAP_FIXED|MAP_DENYWRITE, 3,
 0x182000) = 0x7f34cd9fa000
close(3)                                = 0
openat(AT_FDCWD, "/usr/lib/libc.so.6", O_RDONLY|O_CLOEXEC) = 3
read(3, "\177ELF\2\1\1\3\0\0\0\0\0\0\0\0\3\0>\0\1\0\0\0000C\2\0\0\0\0\0"..., 832) = 832
lseek(3, 792, SEEK_SET)                 = 792
read(3, "\4\0\0\0\24\0\0\0\3\0\0\0GNU\0\201\336\t\36\251c\324\233E\371SoK\5H\334"..., 68)
 = 68
fstat(3, {st_mode=S_IFREG|0755, st_size=2136840, ...}) = 0
lseek(3, 792, SEEK_SET)                 = 792
read(3, "\4\0\0\0\24\0\0\0\3\0\0\0GNU\0\201\336\t\36\251c\324\233E\371SoK\5H\334"..., 68)
 = 68
lseek(3, 864, SEEK_SET)                 = 864
read(3, "\4\0\0\0\20\0\0\0\5\0\0\0GNU\0\2\0\0\300\4\0\0\0\3\0\0\0\0\0\0\0", 32) = 32
mmap(NULL, 1848896, PROT_READ, MAP_PRIVATE|MAP_DENYWRITE, 3, 0) = 0x7f34cd6b3000
mprotect(0x7f34cd6d5000, 1671168, PROT_NONE) = 0
mmap(0x7f34cd6d5000, 1355776, PROT_READ|PROT_EXEC, MAP_PRIVATE|MAP_FIXED|MAP_DENYWRITE,
 3, 0x22000) = 0x7f34cd6d5000
mmap(0x7f34cd820000, 311296, PROT_READ, MAP_PRIVATE|MAP_FIXED|MAP_DENYWRITE, 3, 0x16d000)
 = 0x7f34cd820000
mmap(0x7f34cd86d000, 24576, PROT_READ|PROT_WRITE, MAP_PRIVATE|MAP_FIXED|MAP_DENYWRITE, 3,
 0x1b9000) = 0x7f34cd86d000
mmap(0x7f34cd873000, 13888, PROT_READ|PROT_WRITE, MAP_PRIVATE|MAP_FIXED|MAP_ANONYMOUS, -1,
 0) = 0x7f34cd873000
close(3)                                = 0
mmap(NULL, 12288, PROT_READ|PROT_WRITE, MAP_PRIVATE|MAP_ANONYMOUS, -1, 0) = 0x7f34cd6b0000
arch_prctl(ARCH_SET_FS, 0x7f34cd6b0740) = 0
mprotect(0x7f34cd86d000, 16384, PROT_READ) = 0
mprotect(0x7f34cd9fa000, 4096, PROT_READ) = 0
mprotect(0x7f34cda17000, 4096, PROT_READ) = 0
mprotect(0x557ef80dc000, 4096, PROT_READ) = 0
mprotect(0x7f34cdaa1000, 4096, PROT_READ) = 0
munmap(0x7f34cda1f000, 362838)          = 0
set_tid_address(0x7f34cd6b0a10)         = 9760
set_robust_list(0x7f34cd6b0a20, 24)     = 0
rt_sigaction(SIGRTMIN, {sa_handler=0x7f34cda02130, sa_mask=[],
 sa_flags=SA_RESTORER|SA_SIGINFO, sa_restorer=0x7f34cda0e3c0}, NULL, 8) = 0
rt_sigaction(SIGRT_1, {sa_handler=0x7f34cda021d0, sa_mask=[],
 sa_flags=SA_RESTORER|SA_RESTART|SA_SIGINFO, sa_restorer=0x7f34cda0e3c0}, NULL, 8) = 0
rt_sigprocmask(SIG_UNBLOCK, [RTMIN RT_1], NULL, 8) = 0
prlimit64(0, RLIMIT_STACK, NULL, {rlim_cur=8192*1024, rlim_max=RLIM64_INFINITY}) = 0
brk(NULL)                               = 0x557ef8389000
brk(0x557ef83aa000)                     = 0x557ef83aa000
mmap(NULL, 2101248, PROT_READ|PROT_WRITE, MAP_PRIVATE|MAP_ANONYMOUS, -1, 0) = 0x7f34cd4af000
mmap(NULL, 2101248, PROT_READ|PROT_WRITE, MAP_PRIVATE|MAP_ANONYMOUS, -1, 0) = 0x7f34cd2ae000
clone(child_stack=NULL, flags=CLONE_CHILD_CLEARTID|CLONE_CHILD_SETTID|SIGCHLD, 
child_tidptr=0x7f34cd6b0a10) = 9762
wait4(9762, ./lab01 10
10.12.63	4d4691e7cfebe79b9722e274750b4222b2470c5433e8a15dbaac06b3
0.12.83	4ea299d7ad460e7b355e60fb04b2cd99ff36b36585c6d2d6eaf87f
3.10.125	42efe7b30ef5777e4cd4b1b927f672
15.12.246	4049b4392bc0998998b375
10.10.422	80c751cea9992cd8560624ca306ebb98
11.7.475	59ce7e1d0247467518
6.0.499	1c6b85f4fce7
29.9.1096	a4c44cd3bfb6438db30d88e570bf3d87c942edcae11de5f538de12975fd385
22.7.1321	
2.8.1994	e037ae90587681c36ad18b10028d4b6359
[{WIFEXITED(s) && WEXITSTATUS(s) == 0}], 0, NULL) = 9762
--- SIGCHLD {si_signo=SIGCHLD, si_code=CLD_EXITED, si_pid=9762, si_uid=1000,
 si_status=0, si_utime=0, si_stime=0} ---
exit_group(0)                           = ?
+++ exited with 0 +++

\end{lstlisting}}

При вызове <<strace>> для проверки корректности работы программы была использована лабораторная работа по дискретному анализу, для которой необходим ввод из файла (сортировка значений за линейное время по ключу типа DD.MM.YYYY)

\subsection*{Дневник отладки}  

Для начала я освежил знания по созданию процессов и связи между ними.

Затем производил эксперименты с функцией $exec$ и её производными.
В процессе изучения я понял, что путь для программы необходимо указывать относительно (в случае если команда не содержится в $/usr/bin$), и относительным $./path/to/file$ если она не лежит в $/usr/bin$.

\subsection*{Недочёты}

Задача данной лабораторной заключается именно в получении базовых знаний работы с процессами и системными вызовами. Практической необходимости создания нескольких процессов по моему варианту задания не требуется (за меня всё сделает $exec$)

\subsection*{Выводы}

Вызов $fork$ дублирует породивший его процесс со всеми его переменными, файловыми дескрипторами, приоритетами процесса, рабочий и корневой каталоги, и сегментами выделенной памяти.

Ребёнок {\bf не} наследует:
\begin{itemize}
    \item идентификатора процесса (PID, PPID);
    \item израсходованного времени ЦП (оно обнуляется);
    \item сигналов процесса-родителя, требующих ответа;
    \item блокированных файлов (record locking).
\end{itemize}

В процессе выполнения лабораторной работы были приобретены навыки практического применения создания, обработки и отслеживания их состояния. Для выполнения данного варианта задания создание потоков как таковых не требуется, так как всю работу выполняет системный вызов <<exec>>.

\end{document}

