\documentclass[12pt]{article}

\usepackage[utf8x]{inputenc}
\usepackage[T1, T2A]{fontenc}
\usepackage{fullpage}
\usepackage{multicol,multirow}
\usepackage{tabularx}
\usepackage{ulem}
\usepackage{listings} 
\usepackage[english,russian]{babel}
\usepackage{tikz}
\usepackage{pgfplots}
\usepackage{indentfirst}
\usepackage{ulem} 


\parindent=1cm
\makeatletter
\newcommand{\rindex}[2][\imki@jobname]{%
    \index[#1]{\detokenize{#2}}%
}
\makeatother
\newcolumntype{P}[1]{>{\raggedbottom\arraybackslash}p{#1}}

\linespread{1}
\pgfplotsset{compat=1.16}
\begin{document}

\section*{\centering Лабораторная работа №\,5 по курсу :\\ Операционные системы}

\noindent Выполнил студент группы М8О-208Б-17 МАИ \,\, \textit{Милько Павел}.

\subsection*{Цель работы}

\noindent Целью является приобретение практических навыков в:

\begin{itemize}
    \item Создание динамических библиотек
    \item Создание программ, которые используют функции динамических библиотек
\end{itemize}

\subsection*{Задание}

\noindent Требуется создать динамическую библиотеку, которая реализует определенный функционал. Далее использовать данную библиотеку 2-мя способами:

\begin{enumerate}
    \item Во время компиляции (на этапе «линковки»/linking)
    \item Во время исполнения программы, подгрузив библиотеку в память с помощью системных вызовов
\end{enumerate}

\noindent В конечном итоге, программа должна состоять из следующих частей:

\begin{itemize}
    \item Динамическая библиотека, реализующая заданных вариантом интерфейс;
    \item Тестовая программа, которая используют библиотеку, используя знания полученные на этапе компиляции;
    \item Тестовая программа, которая использует библиотеку, используя только местоположение динамической библиотеки и ее интерфейс.
\end{itemize}

\noindent Структура данных, с которой должна обеспечивать работу библиотека:

 {\bf Работа с деком}

\noindent Тип данных, используемый структурой:

{\bf Вещественный 64-битный}

\subsection*{Информация}

Динамическая библиотека - по своей сути обычный объектный файл, который может присоединяется к другой программе в два этапа. Первый этап, это естественно этап компиляции. На этом этапе линковщик встраивает в программу описания требуемых функций и переменных, которые присутствуют в библиотеке. Сами объектные файлы из библиотеки не присоединяются к программе. Присоединение этих объектных файлов осуществляет системный динамический загрузчик во время запуска программы. Загрузчик проверяет все библиотеки прилинкованные с программе на наличие требуемых объектных файлов, затем загружает их в память и присоединяет их в копии запущенной программы, находящейся в памяти. Это позволяет не забивать память лишними одинаковыми частями программ (наиболее яркий пример -- $libc$, с которой пролинкованны практически все исполняемые файлы).


С созданием библиотеки из исходного кода могут возникнуть некоторые проблемы, а конкретно переход между функциями с помощью операции ассемблера -- $goto$. Так как обычные объектные файлы используют абсолютную адресацию, нужно перенастроить их на относительную, с помощью ключа \lstinline|-fPIC| ({\bf Position Independent Code})

Ну и чтобы получить саму динамическую библиотеку надо указать ключ \lstinline|-shared|.

\subsubsection*{Линковка с динамической библиотекой:}

Для использования полученной библиотеки в других программах нужно указать при линковке флаг \lstinline|-f%name%|. Важно указать именно название библиотеки, а не её имя. То есть если файл называется ``libdeck.so'', то название самой библиотеки -- ``deck'' и подключать её надо с помощью ключа \lstinline|-ldeck|.

Но если сразу запустить полученный исполняемый файл, то появится ошибка:

\begin{lstlisting}
./run: error in loading shared libraries: libdeck.so: cannot open 
shared object file: No such file or directory
\end{lstlisting}

Это происходит потому, что система не знает где ей искать файл с таким именем, она пытается найти её по заранее определённым системным путям (определены в файле ``/etc/ld.so.conf'')

Также в дополнение к этим путям можно использовать переменную окружения: 

\begin{lstlisting}[language=bash]
export LD_LIBRARY_PATH=.
\end{lstlisting}

\subsubsection*{Загрузка библиотеки в память с помощью системных вызовов:}

Использовать динамические библиотеки можно не только в начале загрузки, но и в процессе самой работы программы. Программа сама может вызывать любые функции из библиотеки, когда это необходимо. Это обеспечивает библиотека dl, которая позволяет линковать библиотеки ``на лету''. Она управляет загрузкой динамических библиотек, вызовом функций из них и выгрузкой после конца работы.

Для использования функций работы с динамическими библиотеками необходимо подключить заголовочный файл: 
\lstinline|#include <dlfcn.c>|

Чтобы можно было извлечь какие-либо функции из библиотеки сначала необходимо их загрузить. Этим занимается функция ``dlopen'':
\begin{lstlisting}[language=c]
void *dlopen (const char *filename, int flag);
\end{lstlisting}

За загрузку самих функций из библиотеки отвечает функция

\begin{lstlisting}[language=c]
void *dlsym(void *handle, char *symbol);
\end{lstlisting}

Для этой функции требуется адрес загруженной библиотеки handle, полученный при открытии функцией dlopen(). Требуемая функция или переменная задается своим именем в переменной symbol.
\newline

После окончания работы с библиотекой её следует закрыть:
\begin{lstlisting}[language=c]
dlclose(void *handle);
\end{lstlisting}

\subsection*{Описание программы}

{ \scriptsize
\begin{lstlisting}[language=c]
#include "deck.h"
#include <stdio.h>

#ifdef LIBLOAD
#include <dlfcn.h>
#endif

int main()
{
#ifdef LIBLOAD
    void* library = dlopen("./libdynamic.so", RTLD_LAZY);
    if (!library) {
        fprintf(stderr, "dlopen() error: %s\n", dlerror());
        return 1;
    }
    
    Deck*  (*init_deck) ();
    int    (*push_back) (Deck*, double);
    int    (*push_start)(Deck*, double);
    double (*pop_back)  (Deck*);
    double (*pop_start) (Deck*);
    int    (*clear_deck)(Deck*);
    int    (*empty_deck)(Deck*);
    
    *(void**)(&init_deck)  = dlsym(library, "init");
    *(void**)(&push_back)  = dlsym(library, "pushBack");
    *(void**)(&push_start) = dlsym(library, "pushStart");
    *(void**)(&pop_back)   = dlsym(library, "popBack");
    *(void**)(&pop_start)  = dlsym(library, "popStart");
    *(void**)(&clear_deck) = dlsym(library, "clear");
    *(void**)(&empty_deck) = dlsym(library, "empty");
    
#define init      init_deck
#define pushBack  push_back
#define pushStart push_start
#define popBack   pop_back
#define popStart  pop_start
#define clear     clear_deck
#define empty     empty_deck
#endif // LIBLOAD
    
    Deck* deck = init();
    
    pushBack (deck, 2.0);
    pushStart(deck, 1.0);
    pushBack (deck, 4.0);
    pushStart(deck, -1.0);
    pushBack (deck, 1.0);
    
    printf("%lf ", popStart(deck));
    printf("%lf ", popBack (deck));
    printf("%lf ", popStart(deck));
    printf("%lf ", popBack (deck));
    printf("%lf ", popStart(deck));
    
    pushStart(deck, 19.992818);
    pushBack (deck, 20.992818);
    pushStart(deck, 21.992818);
    pushBack (deck, 22.992818);
    printf("%lf\n", popStart(deck));
    
    clear(deck);
    free(deck);
#ifdef LIBLOAD
    dlclose(library);
#endif
    return 0;
}

\end{lstlisting}}


\subsection*{Работа программы}

\noindent По заданию надо было сделать 2 разные программы, отличающиеся способом подключения функций: через слинкованные файлы и с помощью системных вызовов загрузки библиотек.

\noindent С помощью директив препроцессора я смог уместить обе программы в одном файле.
\newline

\noindent Лог компиляции:
{\scriptsize 
\begin{lstlisting}[language=bash]
gcc -Wall -Wextra -Wpedantic -W -g3 -c -fPIC -o deck.o deck.c
gcc -Wall -Wextra -Wpedantic -W -g3 -c -o main.o main.c
gcc -Wall -Wextra -Wpedantic -W -g3 -shared -fPIC -o libdynamic.so deck.o
gcc -Wall -Wextra -Wpedantic -W -g3 -ldl -DLIBLOAD main.c -o load
gcc -Wall -Wextra -Wpedantic -W -g3 -L. -ldynamic -o linked main.o
\end{lstlisting}
}

\noindent То есть на выходе получились 2 файла (``linked'' и ``load'') и легко проверить их различие с помощью утилиты $ldd$:

{\scriptsize 
\begin{lstlisting}[language=bash]
/home/Desktop/OC/lab05 $ ldd linked
	linux-vdso.so.1 (0x00007ffdeab83000)
	libdynamic.so => ./libdynamic.so (0x00007fd40a1dd000)
	libc.so.6 => /usr/lib/libc.so.6 (0x00007fd409fba000)
	/lib64/ld-linux-x86-64.so.2 => /usr/lib64/ld-linux-x86-64.so.2 (0x00007fd40a1e9000)

/home/Desktop/OC/lab05 $ ldd load
	linux-vdso.so.1 (0x00007ffe5610e000)
	libdl.so.2 => /usr/lib/libdl.so.2 (0x00007fb98e904000)
	libc.so.6 => /usr/lib/libc.so.6 (0x00007fb98e740000)
	/lib64/ld-linux-x86-64.so.2 => /usr/lib64/ld-linux-x86-64.so.2 (0x00007fb98e96f000)
\end{lstlisting}
}

\noindent И вывод $strace$ подтверждает это ещё раз:

{\scriptsize 
\begin{lstlisting}[escapechar=!]
/home/Desktop/OC/lab05 $ strace -e trace=openat ./linked 
openat(AT_FDCWD, "./libdynamic.so", O_RDONLY|O_CLOEXEC) = 3
openat(AT_FDCWD, "/etc/ld.so.cache", O_RDONLY|O_CLOEXEC) = 3
openat(AT_FDCWD, "/usr/lib/libc.so.6", O_RDONLY|O_CLOEXEC) = 3
-1.000000 1.000000 1.000000 4.000000 2.000000 21.992818
+++ exited with 0 +++

/home/Desktop/OC/lab05 $ strace -e trace=openat ./load
openat(AT_FDCWD, "/etc/ld.so.cache", O_RDONLY|O_CLOEXEC) = 3
openat(AT_FDCWD, "/usr/lib/libdl.so.2", O_RDONLY|O_CLOEXEC) = 3
openat(AT_FDCWD, "/usr/lib/libc.so.6", O_RDONLY|O_CLOEXEC) = 3
openat(AT_FDCWD, "./libdynamic.so", O_RDONLY|O_CLOEXEC) = 3
-1.000000 1.000000 1.000000 4.000000 2.000000 21.992818
+++ exited with 0 +++
\end{lstlisting}}


\subsection*{Дневник отладки}  


\subsection*{Выводы}



В который раз были написаны интерфейс и реализация одного из часто используемых АТД, в моем случае дека. Никаких особых сложностей не возникло, всё заработало с первого раза и так, как надо. Не вызвала проблем и сборка динамической библиотеки. При первом запуске слинкованной программы естественно вылезла ошибка, так как я забыл про  \lstinline|export LD_LIBRARY_PATH=.|. Работа с подгружаемой в память библиотекой так же оказалась не слишком сложной (громоздкой и некрасивой да, но не сложной)

Статическая линковка удобна тем, что собирает программу и рантайм в один файл. После запуска программы, реализация используемых функций ищется в сборке, таким образом, гарантируется переносимость программы. Как результат -- сборка несколько увеличивается в размерах. При динамической линковке мы получаем ``голую'' сборку без сторонних библиотек. Ее размер, безусловно, меньше, но при этом мы должны гарантировать, что на клиентской машине имеется библиотека, используемая в программе, и ее версия одинакова с той, которая была использована при сборке. В обоих способах есть минусы и плюсы, выбор зависит от необходимого результата.


\end{document}

